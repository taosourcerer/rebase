\documentclass[a4paper,hidelinks]{article}
\usepackage{graphicx}
%\usepackage[margin=1.5in]{geometry}
\usepackage{geometry}
\usepackage{xcolor}

\begin{document}

\newgeometry{top=15mm, bottom=0mm}

~
\vspace{55px}

\begin{center}
{\Huge \textbf{
Rebase
}}\\
\vspace{10px}
\textit{
A crystallization manual for the melting mind
}
\end{center}

\vspace{70px}

\begin{center}
\makebox[\textwidth]{\includegraphics[width=1.45\textwidth]{math_tile_deformation.jpg}}
\end{center}

\thispagestyle{empty}

\newpage

\vspace*{\fill}
Everything you believed in turned out to be false. The direction your life was going in turned out to be the opposite of what you wanted. Your life become chaos. It's time to rebase.
\vspace*{\fill}

\restoregeometry
\newgeometry{margin=1.75in}

\pagenumbering{arabic}

\begin{center}
\textit{
Screw you guys, I'm going home.  - Eric Cartman
}
\end{center}

\textbf{The physical is stronger than the mental}.
When someone punches you in the face you will feel pain, no matter how strongly you say \textit{I feel no pain}.
The elimination of negative emotion is more important than creation of positive emotion.
Wounds can't heal if you're standing in the middle of the battlefield.
If the problem is your job, quit your job and find a new one.
If the problem is your ex, stay away from them.
If someone's company is making you feel bad, stay away from them.
Do not share personal details of your life with others, except close friends and family.
Reduce your attack surface.
Remove people who create stress form your life.
Social media causes pain, do not use it.
Your negative mood will chase away potential romantic partners, and their rejections are only going to make you feel even worse.
Even if you have some luck finding someone you will fuck it up because of bad mood and negativity.
Stay away from romantic relationships of any kind.
Stay away from things which make you feel bad.

\newpage

\begin{center}
\textit{
I let go.  Lost in oblivion.  Dark and silent and complete.  I found freedom.  Losing all hope was freedom. - Chuck Palahniuk
}
\end{center}

\textbf{There is no need to do anything}.
There is freedom, to do anything you wish.
Thoughts of things which \textit{should} be done are just social conditioning talking.
There are no rules for life other than eat, sleep, breathe, and survive.
The rest is made up.
It doesn't matter.
Remove everything form your todo list.
Literally delete it.
Remove all goals form your life.
Planing is not needed.
Live in the moment.
Surrender.
Stop making yourself do things.
Let things happen naturally.
Remove every effort.
Say yes.
To all that happens, say yes.
Are things to be taken seriously?
Let it go.
There is nothing you should do.
Relax.
Every thing you feel bad about, you can also laugh about.
Play for fun, not to win.

\newpage

\begin{center}
\textit{
Realise the fact that you simply "live" and not "live for". - Bruce Lee
}
\end{center}

\textbf{Only do things which you enjoy}.
You are a free man.
Concerts are there to be attended.
Movies are there to be watched.
Other people are there to be talked to.
Those activities can fill a life, and that is completely fine.
Every day, buy yourself a present.
That way of life is not in any way inferior than anything else.
Just be careful not to get addicted to drugs or alcohol, of course.
Your life is your life.
Do things for yourself.
Enjoy life.
Live free or die.

\newpage

\begin{center}
\textit{
No plough stops for the dying man.
}
\end{center}

\textbf{The opinions of other people are irrelevant}.
Other people don't even spend time thinking about you.
Outside of close friends and family, nobody cares about you.
Notice when you start worrying about what others think.
Notice it and and stop giving a fuck.
What do you think?
Only that is important.
If a thing is cool to you, do it.
If others think badly of you because of who you are, fuck them.
You are not weird.
You are normal.
It is possible to be ok with people not liking you.
If you hurt other people only because had expectations of you, that's completely their problem.
Even if those people are your parents.
Instead, if they had an agreement with you, only then is it your problem.
Stop pleasing others.
You failed at something, so what?
It doesn't count.
Life is not a game.
You can't win or lose.
The same way you can't win or lose a dance.
The main thing people think about are other people.
They compare themselves with them constantly.
That produces complexes of lower value.
Then people start forcing themselves to become other people which they do not want to be.
Everyone is copying each other's desires all the time, instead of following their own.
This leads to stupid competitions over stupid prizes.
Think for yourself.

\newpage

\begin{center}
\textit{
I know that I know nothing - Socrates
}
\end{center}

\textbf{Forget everything you think you know}.
In order to learn you need to unlearn.
You got into chaos because your model of the world was wrong.
Even if your model of the world is almost completely correct, there are cracks in it.
There are imperfections, and going into those blind spots, those unknown unknowns...
You can acquire knowledge that would completely change your view of things.
The map is not the territory.
Drop your assumptions.
Your interpretation or reality is not reality.
You are not obliged to believe your negative thoughts.
The problems which you think you have are an illusion.
The process of healing is the process of disintegrating of illusions.
The feeling of loss is an illusion.
It is ok that you feel that way.
There is nothing to be gained from ignoring it or trying to block it.
Still, keep in mind that it's an illusion.
You never lost anything because you never had anything.
You never even needed anything.
The feeling of having no control over your life.
The feeling of being weak.
They are all illusions.
The main barrier to learning are unnecessary aversions.
Those aversions are learned structures in the brain, cached responses.
The process which creates them is not perfectly rational.
Clear your cache.
Life goes on, and the truth changes.
What was once true is often no longer true just a little while later.
To grow, people have to let go of the principles and standards with which they define themselves.
You are not who you thought you were.
Being wrong feels the same way like being right does.
Remember, the mind doesn't like to change itself in the face of new information.

\newpage

\begin{center}
\textit{
Drop all your preconceived and fixed ideas and be neutral. Do you know why the cup is useful? Because it is empty. - Bruce Lee
}
\end{center}

\textbf{It's good to be aware of your assumptions}.
When events happen, we form an interpretation of them.
Of what people meant when they said something.
Of what is going to happen.
Of what people think about various things.
All of those interpretations are an illusion.
The truth is not what you think it is.
As you live, your interpretations accumulate.
Keep your mind clean of garbage interpretations.
Think step by step.
Pay close attention to what you really know and what is an assumption.
Create a great number of alternative hypotheses for why something happened.
When you feel something is true, ask yourself, is perhaps the opposite true?
Relax your expectations when going into things.
You do not actually know if things are going to go well or not.
When talking, talk about facts. 
Report what you see as the current state of things. 
One part of the state.
Then another.
Instead of proposing action, look at what’s happening.

\newpage

\begin{center}
\textit{
We are not special. We are not crap or trash, either. We just are. We just are, and what happens just happens. - Chuck Palahniuk
}
\end{center}

\textbf{There are no wrong feelings}.
It's as if someone put a match on you and you feel hot.
Of course you feel that way.
You are free to feel that way, that's ok, don't worry about it.
And if you say further \textit{but I can't help worrying about it}, then ok, worry about it.
Go along with it.
You don't know what you're supposed to do.
What can you do?
If you don't know what you're supposed to do, you watch.
Watch not only what's going on on the outside but also what's going on on the inside.
Treat your own thoughts, reactions, and emotions as if those inside reactions were also outside.
And you're just watching them.
Without attempting to change it in any way.
Without judging it.
Without calling it good or bad.
The subconscious part of the brain and the conscious part wish to communicate.
The rational part of you can't hold the emotional part in submission.
This break in communication comes from the rigid idea that things \textit{should} be a certain way, that you \textit{should} feel this or the other.
Drop your assumptions.
The subconscious needs to express itself.
Blocking your emotions blocks learning.

\newpage

\textbf{You are not alone}.
There are a lot of similar people to you out there, as there have been before in history.
They walked the same path you walk.
They had the same thoughts you have.
Good people are out there.
The world is a large place.
Everything could be ok in the end.
Not all is lost.

\newpage

\begin{center}
\textit{
I say never be complete, I say stop being perfect, I say lets evolve, let the chips fall where they may. - Chuck Palahniuk
}
\end{center}

\textbf{You will never be perfect}.
You have done something you think you shouldn't have.
You have failed to do something you think you should have.
This does not make you a bad, evil, or immoral person.
It makes you a normal person.
The event which occurred is not completely your fault.
You had your share of influence and you will be careful to avoid such behavior in the future.
Your influence was not so large as you think.
The feeling of guilt comes out of projection.
You feel terrible, so you assume the person who you wronged feels as terrible as you do.
In fact, that person feels completely differently.
That person has a lot of other things in their life.
Those things make them feel very different than you do, in a lot of ways.
You think you should have done differently, because a perfect person would do differently.
That is not in line with reality.
You're not supposed to be some imaginary perfect person.
You're supposed to be you.
You could not predict all of the outcomes of your actions.
You did the best you could given the situation and the knowledge you had at that moment.
Do you know why you did what you did?
Understanding leads to forgiveness.
Forgiveness allows you to learn from the past.
You don't need to self-improve.
You are good enough.
Perfection is not attainable.
Everything that will happen, must happen, can't not happen.
The concept of \textit{mistake} is not a valid abstraction to evaluate acitons while embedded in a physical universe.
You can't make a mistake.

\newpage

\begin{center}
\textit{
First you have to give up. First you have to know, not fear, know - that some day you’re gonna die. - Chuck Palahniuk
}
\end{center}

\textbf{Everything is exactly as it should be}.
You are not a victim.
You can't blame others for how you feel.
You are not responsible for how other people feel.
That is their responsibility.
You have your own life.
You have your own path.
Whatever happened to you, you have personally contributed to it.
And that was not a mistake.
You don't need anything from other people.
You don't \textit{deserve} anything.
It's also not the case that you do not \textit{deserve} something.
It has nothing to do with deserving.
Complaining is useless.
You can be grateful for the opportunity to start over from scratch.
You think others are underestimating you, this fuels anger.
They are estimating the best they can, given how little they see.
Your work is boring so you can be interesting when you're not working.
Work but do not control.
Create but do not possess.
Succeed but do not dwell on success.
Achieve without arrogance.
Rise without domination.
Yield and remain whole.
Bend and remain straight.
Be worn out but become renewed.
Have little and receive.
Those who praise themselves have no merit.
Those who boast about themselves do not last.
Those who stand on tiptoes do not stand firmly.
Those who rush ahead don't get very far.
Those who try to outshine others dim thier own light.
Return to the state of the infant.
A good commander achieves result, then stops.
And does not dare to reach for domination.
Achieves result but does not brag.
Achieves result but is not arrogant.
Achieves result but only out of necessity.
Achieves result but does not dominate.
The ultimate honor is no honor.
Do not wish to be shiny like jade, be dull like rocks.
Close the mouth.
Shut the doors.
Blunt the sharpness.
Unravel the knots.
Dim the glare.
Mix the dust.
Learn to laugh at yourself.
Remember that you are going to die.
In fact, you are dying all the time.
You have nothing to lose.
You are already naked.
There is no reason not to follow your heart.

\newpage

\begin{center}
\textit{
The fundamental delusion: there is something out there that will make me happy and fulfilled forever. - Naval Ravikant
}
\end{center}

\textbf{The only thing which exists is now}.
The future and the past are just mental constructs, they are illusions.
Thoughts of what happened in the past.
And how it could have went better than it did.
Or, how it was better than the state we are in now.
Thoughts about pursuing a future.
Thoughts of making one's happiness depend on something which isn't here at all.
When you fulfill your desire, another one will take its place.
Nothing will make you happy forever.
There is no singular turning point in your life from which on everything will be different.
Other people will not save you, nor will they make you happy forever.
The present is the only place you live in.
Memory is a tool.
The reason we have memory is to learn from experience.
If you remember that something bad happened, and you can figure out why, then you can try to avoid that bad thing happening again.
That’s the purpose of memory.
It’s not \textit{to remember the past}.
It’s to stop the same thing from happening over and over.
What we learn are models.
Once we have the models the training data can be discarded.
Look deep within you and you will realize you simply don't \textit{care} about the past.

\newpage

\begin{center}
\textit{
Sometimes I go about in pity for myself, and all the while, a great wind carries me across the sky. - Ojibwe saying
}
\end{center}

\textbf{Look at yourself from the outside perspective}.
Imagine how your whole life would look to your brother, if you had one.
All the obstacles you had, and you overcame.
All the times that people were treating you badly.
How you were in pain.
How you suffered.
Isn't it natural that you felt angry?
Pause and say to yourself
\textit{
Here I am.
I am in a difficult situation.
I am struggling and I am in pain.
}
And then, react to this information just as if we were hearing it from a close friend.
Say, now to yourself,
\textit{
I am so sorry.
I want to care for you in the best way that I can.
How can I help you? What can I do for you?
}
And then we must answer these questions for ourselves.
If a member of your family had the same problems you have you would hug them.
You would be gentle to them, kind, warm.
You may think you hate yourself.
On some deep level, you do.
Going even deeper you will see, underneath it all, underneath the sea of disappointment, anger, horror and sadness, if you look hard enough, you will find that you love yourself.
You want good things to happen to you.
You want yourself to be at peace.
Say to yourself:
\textit{
May I accept my anger, fear, and sadness, knowing that my heart is not limited by them.
May I accept my pain, without thinking it makes me bad or wrong.
May I be at ease and happy.
}

\newpage

\begin{center}
\textit{
I am a human being, therefore nothing human is foreign to me. - Terence
}
\end{center}

\noindent
\textbf{You are an animal}.\\
Inside of us all there is potential for the worst possible behavior.\\
Our minds are evolved.\\
There are parts of our brains which are primitive.\\
Since we are born, we are selfish.\\
Our emotions are often nonsense.\\
Our brain is full of lies.\\
Our ability to reason is very limited.\\
Yet, there is a flame within us.\\
Our heaven is a selfless union with another.\\
We wish to know why the stars shine.\\
We are ugliness, seeking beauty.\\
Imperfections can be endearing.\\
Not just with romantic parthers but with everyone.\\
Not just with people but with art and work.\\
Not just with others but with yourself.\\
I love the fact that everyone, including myself, is a fool.\\

\newpage

\begin{center}
\textit{
The most terrifying thing is to accept oneself completely. - C.G. Jung
}
\end{center}

\textbf{Stop trying to be someone else}.
Let go of who you think you should be.
Let go of what others think you should be.
Let your heart and your intuition guide you.
As you do that, the real you emerges.
Sink into who you are.
In vulnerability, there is strength.
When you think something is wrong with you, you turn to the outside world to tell you what is wrong.
They are happy to provide you with bullshit about what you should change.
Instead, look within at who you are, and accept that.
Learn to stop worrying and love evolution.
You already are the best version of yourself.
It's not better to act than to be still.
It's not better to be \textit{productive} than to do nothing.
It's not better to be disciplined than it is to be relaxed.
All of these are equally valid and good.
Life is not something that can be wasted.
In the same way you can't waste a dance.
You can't waste a song.
Walk with who you are.
Ego is a good thing, it is condensed experience, processed and compressed information.
There is nothing wrong with you.
Happiness is not something you can \textit{not deserve}, it does not have anything to do with deserving.
This is something you can internalize on a deep level.
Meditate on this.
On every breath in slowly and gently say to yourself \textit{I love myself}.
On every breath out just notice whatever is there, without judgment or attachment.

\newpage

\begin{center}
\textit{I believe that it is better to tell the truth than to lie. I believe that it is better to be free than to be a slave. And I believe that it is better to know than be ignorant. - H. L. Mencken}
\end{center}

\textbf{Open up to close friends and family}.
Share your thoughts with someone close.
Looking the world through the eyes of others will make you learn to view your life in a new light.
When you feel an urge to say something to a family member or a close friend, don't be afraid.
It is better to speak than not to speak.
Lying it is the major source of all human stress.
Lying kills people.
The kind of lying that is most deadly is withholding, or keeping back information from someone we think would be affected by it.
Keeping secrets and hiding from other people is a trap.
Never lie, be honest with everyone.
Even white lies are toxic.
Say what you think, speak the truth.
What can be destroyed by truth should be destroyed by truth.
What is true is already so.
Owning up to it doesn't make it worse.
Not being open about it doesn't make it go away.

\newpage

\begin{center}
\textit{If I commit to doing something, then I commit to doing it right now. - Naval Ravikant}
\end{center}

\textbf{Make your life a tiny bit easier and more enjoyable}.
Day by day, step by step.
If something bothers you, fix it.
Do things which are easy to do.
Learn to love to wash the dishes.
Enjoy the way the water feels on your hands.
Pay close attention to the colors, you may notice how beautiful they are.
Play with the task.
Learn to love it.
When doing tasks, don't do them out of duty, responsibility, or obligation.
Don't work on them.
The \textit{real} task may be be \textit{worked on} in the future.
Still, you could check out the situation now, and make some preparations.
You're not working on it, you are preparing it.
Knowledge of the future is impossible.
The only thing you can commit to is a very short period of time, maybe for the next 1 minute.
In reality, the planning horizon is always very short.
Improve small everyday things.

\newpage

\begin{center}
\textit{Only after disaster can we be resurrected. It's only after you've lost everything that you're free to do anything. Nothing is static, everything is evolving, everything is falling apart. - Chuck Palahniuk}
\end{center}

\textbf{Discover something you enjoy doing}.
We need a lot of downtime.
A lot of daydreaming time.
A lot of slack and unstructured time.
We need boredom.
We need to be out of the limelight for long periods of time.
Only failure gives you that kind of alone isolation downtime where you can find yourself.
Everyone spends so much time looking around themselves.
Almost no one is looking up.
Watch inspiring movies, something which represents your ideals and principles.
Curiosity, honesty, creativity.
Listen to inspiring music.
The world is an interesting place which you can study.
Go outside, walk around, go into nature.
You can do things just for fun, just to see what will happen.
Follow your curiosity.
Approach things like a scientist, do things as an experiment.
Experiments can be done even if you are feeling unwell.
Moving in different directions, you will see what you like.
Watch in which direction you are naturally flowing.
Let your intuition guide you.
Take feedback from your environment.
When it comes to your life, gardening works better than engineering.
You can produce something, write, take a photo...
Not because you have to, but just for fun.
Mess things up.
Do them poorly.
Improvise.
Do silly, unconventional, spontanous and irrational things.
It needs to be pure fun.
If you like swimming, you don't swim to get to the other side.
You do it to move, to feel the water.
What would you do if you had an infinite amount of resources and and infinite amount of time?
Whatever is within you, express it.
Release it.
Do whatever \textit{you} intrinsically \textit{want} to do.

\newpage

\begin{center}
\textit{School, politics, sports, and games train us to compete against others. True rewards such as knowledge, love, and equanimity, they come from ignoring others. - Naval Ravikant}
\end{center}

\textbf{Ignore irrelevant things}.
Take one thing you care about the most and run with it.
Go all the way.
Remove all barriers.
Give zero fucks about other things.
Caring about many things at the same time just creates anxiety.
You can change what you care about later.
Now focus on what is the most important thing is this phase of your life.
You need to feel the pull of the thing calling you to it.
When you are thirsty, it takes no energy to drink water.
It takes energy \textit{not} to drink water.
Pushing yourself it is not needed.
About other things, things you don't give a fuck about, be spontaneous.
Intelligence is learning what you can ignore.

\newpage

\begin{center}
\textit{Our business is not to see what lies dimly at a distance, but to do what lies clearly at hand. - Bruce Lee}
\end{center}

\textbf{Your every action should be beneficial in and of itself}.
Do not plan anything.
Solving beats planning.
The shortest path between two points is a line.
Think short term.
Gradient following works better than discipline.
Things which look hard or boring from a distance, when you come closer and apply action, get transformed.
Do more, think less.
You learn by doing.
To become good at something a surprisingly large amount of repetition is needed, practice, experience.
Move step by step.
What is needed is a lot of specific knowledge of the details, doing things which don't generalize, which don't scale, rote memorization.
Ignore theory.
Experience is the only way to learn.
Instead of reading, write.
Always go with the line of the least resistance.
Act automatically, without thinking, without effort.
When you feel inspired, do it, don't pause too long.
Inspiration is perishable.
Use it while you have it.
While you don't have it, be patient.
Aim low.
Move slowly.
Silent, like water.

\newpage

\begin{center}
\textit{Discipline is just you fighting with yourself to do something you don't want to do. Find things that excite you. - Naval Ravikant}
\end{center}

\textbf{A horse which loves the track runs faster}.
That is a horse which, when exhausted, stops running.
When one part of you wants to do something and another part is resisting, perform a cost-benefit analysis.
How would the world look like if you did do it?
How would it look like if you didn't?
Do you really want to do it?
There is no need for pain.
It could be that you actually don't want to do it.
In that case, don't do it.
At the same time, don't let fear of failure stop you.
What would happen if you failed?
You would gain the most important thing, knowledge.
Failure is the consequence of a lack of knowledge.
This lack of knowledge is the result of a lack of experience.
Failure is a a better teacher than success.
If you decide you want to do it, fall towards it with no effort.
No struggle.
Do it slowly and in a relaxed manner.
Casually.

\newpage

\begin{center}
\textit{Nothing that we do lasts. Eventually, you will fade. Your works will fade. Your children will fade. Your thoughts will fade. This planet will fade. The sun will fade. It will all be gone. - Naval Ravikant}
\end{center}

\textbf{Every endeavor is a sequence of small tasks}.
You can complete them one step at a time.
The only thing to do now is to do the first task, if you want.
Ignore the other tasks.
Take small steps.
You may think you have not been doing enough.
This does not make you a lazy person or a procrastinator.
In fact, you have been doing exactly as much as you should have.
You have been doing enough.
When you fail, it doesn't mean anything.
It's just an event that happened.
Sometimes you fail, sometimes you succeed.
Either way, it does not last forever.
It's not actually important.
Life is a vast ocean of neverending tasks.
Play for fun, not to win.
Plan difficult tasks through the simplest tasks.
Achieve large tasks through the smallest tasks.
The difficult tasks of the world must be handled through the simple tasks.
The large tasks of the world must be handled through the small tasks.
Therefore, sages never attempt great deeds all through life.

\newpage

\begin{center}
\textit{Empty your mind, be formless, shapeless, like water. If you put water into a cup, it becomes the cup. You put water into a bottle and it becomes the bottle. You put it in a teapot it becomes the teapot. - Bruce Lee}
\end{center}

\textbf{Instead of addictions, be useless}.
News, youtube, porn, gaming, reddit, facebook, drinking, smoking, working overtime, empty sex, getting into friends with benefits, status competitions, obsession with safety, chasing money, power, fame, popularity.
Sometimes you do those things because you want to do it.
That is ok.
Sometimes you do it because you feel bored, or you feel anxious, or you are addicted to them.
Sometimes you do it automatically without even thinking.
Do you enjoy performing some activity you are addicted to?
Probably not.
Those activities don't have any value.
They also block you from being creative.
Addictions are hijacking your brain and taking your freedom away.
In those cases try to notice it and instead meditate, or do nothing.
Instead of trying to discipline yourself into avoiding temptations, remove the temptations from your environment.
Watching youtube is a bad way to relax because you still need to be engaged.
Taking a walk is better than being addicted.
The best meditation is doing nothing.
The tree which is useless for wood carving is the one which survives and offers shade.
The best ideas sometimes come in the shower.
Being bored gives you time to think.
To wander into uncharted territory.
To think about what's happening.
Lets curiosity develop.
Gives you ideas for experimentation.
Gives you time to flow.

\newpage

\begin{center}
\textit{Much unhappiness has come into the world because of bewilderment and things left unsaid.}
\end{center}

\textbf{Choose communication}.
Emotional control is hard.
Sometimes changing the way you think can change what you feel.
If it's a change from a negative emotion to a positive, that's great.
Some other times, changing the way you think will not change what you feel.
Sometimes the original emotion is still there.
Even though you say you shouldn't feel that way.
When it comes to dealing with other people, emotional control is most often impossible.
You can be aware that it's irrational to feel some way and still feel that way.
You think you should feel some way but you don't feel that way.
In that case, just feel whatever it is that you feel.
Don't try to override the emotion.
Repressing emotions often just hides them, it does not destroy them.
They are there in the background, looking for justifiable ways to express themselves, something they can latch on to.
Then they slowly start coming out in various small ways, over the years.
That what is repressed the most is the dark side, your shadow, the ugly emotions, which we are afraid to even experience them.
Be honest to yourself about what you're feeling, instead of trying to feel the right thing or to ignore the bad feelings.
There are parts of yourself you wish were not there, the emotions and thoughts you just want to bury.
Allow yourself to feel those emotions in their full intensity.
Communicate those emotions to other people.
Create some environment in which you can be completely honest to others.
Through communication, your feelings will change.
The first step is describing the facts, what has happened.
Without judging the other person.
What do you see?
The second step is, explain how that made you feel.
The third step is to say your idea of an alternative way things could be done.
That may even result in an argument.
Sometimes you need to argue with people.
Otherwise, they don't listen.
Otherwise, they don't realize it's important.
In a similar way like children sometimes need to have a fight.

\newpage

\begin{center}
\textit{Stand up straight with your shoulders back. - Jordan Peterson}
\end{center}

\textbf{Look people in the eyes}.
People naturally start to push boundaries, the same way children do.
When someone is treating you badly, be honest and tell them what is bothering you.
Politeness towards others is proper even when they are rude to you.
Refuse to be pushed around.
No one is allowed to fuck with you.
You are not to blame.
You are not the problem.
Ask them is everything ok.
Ask them if they have a problem with you.
Others may try to make you think you are crazy.
You are not.
Others are not allowed to treat you like you're a loser.
Piling up resentment will push you to the dark side.
Your dark side is a serious matter.
When someone is doing something you don't like, tell them.
Tell the truth and let the chips fall where they may.
Dare to be dangerous, in a controlled way.
So that you don't become dangerous in an uncontrolled way.
Stop pleasing others.
Unpleasant direct communication is needed.
Remember you know exactly what you are doing.
You are gaining knowledge.
Stand up straight.
Speak your mind.

\newpage

\begin{center}
\textit{The death-grip with which one holds on to principles is a source of unhappiness and anger. - Brad Blantnon}
\end{center}

\textbf{Anger is the result of incorrect predictions}.
When it comes to being angry at inanimate objects, the acts of nature, at \textit{fortuna}, think about the following.
If you knew this was coming all along, you would not be angry.
When it comes to being angry at a person, there are two ways you can stop being angry at them.
The first one is talking to them.
If you are angry with a family member, close friend, or a coworker, talk to them.
Anger should be expressed.
Any anger that is not coming out, flowing freely, will turn into sadism.
The second way to stop being angry with someone is to forget they ever existed.
That's possible when it comes to strangers.
In that case, think about the following...
That person is not a piece of shit.
Every human being is a combination of positive and negative attributes.
You invent motives that explain why the other person did what he or she did.
These hypotheses are often wrong.
You forget to question what you are saying to yourself.
Ask yourself did the other person have the right to act in such a way.
In most cases they did have the right.
It's a free country.
You think it's not fair.
You think the way things are is not justice.
You think things should be different.
Well, they shouldn't.
The universe is as it is.
It's your picture of the universe which is wrong.
Your expectations were wrong.
If you were omniscient you would never get angry.
The universe is not here for you.
The universe doesn't think in terms of who \textit{deserves} or does not \textit{deserve} anything.
Once you have thought about these things, if that does not make you less angry, be angry if you wish so.

\newpage

% When someone is criticizing or attacking you, his motives may be to help you or to hurt you. What the critic says may be right or wrong, or somewhere in between. But it is not wise to focus on these issues initially. Instead, ask the person a series of specific questions designed to find out exactly what he means. Try to avoid being judgemental or defensive as you ask the questions. Constantly ask for more and more specific information. Attempt to see the world through the critic's eyes. If the person attacks you with vague, insulting labels, ask him or her to be more specific and to point out exactly what it is about you the person dislikes. This initial manoeuvre can itself go a long way to getting the critic off your back, and will help transform an attack-defense interaction into one of collaboration and mutual respect. If someone is shooting at you, you have three choices: You can stand and shoot back - this usually leads to warfare and mutual destruction; you can run away or try to dodge the bullets - this often results in humiliation and a loss of self-esteem; or you can stay put and skilfully disarm your opponent. I have found that this third solution is by far the most satisfying. When you take the wind out of the other person's sails, you end up the winner, and your opponent more often than not will also feel like a winner. How is this accomplished? It's simple: Whether your critic is right or wrong, initially find some way to agree with him or her. Now, suppose the person who's attacking you is making criticisms you feel are unfair and not valid. What if it would be unrealistic for you to change? How can you agree with someone when you feel certain that what is being said is utter nonsense? It's easy - you can agree in principle with the criticism, or you can find some grain of truth in the statement and agree with that, or you can acknowledge that the person's upset is understandable because it is based on how he or she views the situation. Once you have listened to your critic, found some way to agree with him, you will then be in a position to explain your position and emotions tactfully but assertively, and to negotiate any real differences. Let's assume that the critic is just plain wrong. How can you express this in a non-destructive manner? This is simple: You can express your point of view objectively with an acknowledgement you might be wrong. Make the conflict one based on fact rather than personality or pride. In many cases you will be just plain wrong, and the critic will be right. In such a situation your critic's respect for you will probably increase by an orbital jump if you assertively agree with the criticism, thank the person for providing you with the information, and apologize for any hurt you might have caused. It sounds like old-fashioned common sense (and it is), but it can be amazingly effective.

% don't let resentment build up
% hating others leads to hating yourself
% judging others leads to judging yourself.
% the work by katie byron?
% if you "cross someone off", after some time, reconsider that decision
% maybe it was taken in a moment of clouded judgement
% it's never too late to forgive

\begin{center}
\textit{You must have chaos within you to give birth to a dancing star. - Friedrich Nietzsche.}
\end{center}

\textbf{Remember what you value}.
Truth.
Beauty.
Take a cosmic perspective.
Let your principles and ideals be at the forefront of your mind.
Your ideals are the lighthouse.
In that way, you can truly be independent and not care about the opinions of others.
You need something you care more about than other people.
Being angry or feeling concerned about some things is just a waste of time.
Keeping your eye on your ideals gives you the power to ignore anger and anxiety.
What is constantly tugging at you?
It is not actually a matter of what is important, or what is productive.
It's a question of do you wish to spend time like that.
You have a choice what to think about.
What you pay attention to.
You decide what has meaning and what doesn’t.
There are two fundamental ways of being: flow and discomfort.
What's obstructing your flow my friend?
Frustration, anxiety and cravings are mind-hacks trying to take over your attention.
There are two transitions, out of flow and into flow.
Out of flow transition cannot be controlled since when you are in flow you are lost in it.
Into flow transition can be helped by paying attention to what you are thinking about and then deciding what to do about the thought.
You may decide to write it down for later.
Or decide that it's not worth thinking about at all.
What do you want to pay attention to?
Once you know that, look at the thing, focus on the details.
Focus on your breath, without controlling it.
Focus on the feeling of touch.
Focus on the sounds you hear.
In that way you can flow towards your ideals.

\newpage

\begin{center}
\textit{Ordinary men hate solitude. But the Master makes use of it, embracing his aloneness, realizing he is one with the whole universe. - Lao Tzu}
\end{center}

\textbf{Isolation is a gift}.
We are all alone.
Born alone.
Die alone.
We shall all someday look back on our lives and see that, in spite of our company, we were alone the whole way.
Not lonely, but essentially, and finally, alone.
This is what makes your self-respect so important.
There is no way you can respect yourself if you must look in the hearts and minds of others for your happiness.
In solitude, the nameless can be found.
The most profound experiences can only be had alone.
The Tao which can be named is not the eternal Tao.
We seek beauty and truth, which can't be fulfilled by others.
Trying to do so results in a godlike idealizeation of the partner and dependence on them for our self-worth.
Nobody is going to save you.
Only alone you can find yourself.
That is the time to be creative.
Solitude is for the mind as food is for the body.

\newpage

\begin{center}
\textit{If you're not doing what you want, if you're not earning, you're not learning, what the fuck are you doing? Don't spend time making other people happy, other people being happy is their problem, it's not your problem. - Naval Ravikant.}
\end{center}

\textbf{You're doing it for yourself}.
Remember that you're not doing it for others.
You are personally responsible for your continued existence.
You need to orient yourself.
You need to find a path for yourself.
You seek knowledge for yourself.
The man is an end into himself.
Live for your own sake.
Remember that the greatest artists created art for themselves, not for others.
The path of sacrifice for others leads to madness.
To really experience your life, the focus must be on you as the center of the perceived universe.
Any time that you are doing something that you think is not for you, examine both your thinking and your actions.
If it isn't for you, you’re doing it wrong.
Every time you make a decision, think of yourself first.
Only after that think of others.
Ask for what you want.
There are only two risks attached.
The first is that asking will let both you and others know who and where you are.
That is necessary for learning your real identity.
The second risk in asking for what you want is that you might get it.

\newpage

\begin{center}
\textit{What does not kill me makes me stronger - Friedrich Nietzsche}
\end{center}

\textbf{Take small amounts of damage, slowly}.
Sometimes, take a cold shower.
Sometimes, lift weights.
Take controlled small amounts of damage and pain.
Perform rejection therapy.
The world revelas itself to those who travel on foot.
Read.
Test your limits.
Exposing yourself to small amounts of displeasure builds tolerance for the larger amounts.
Go gradually.
Slowly move out of your comfort zone.
You will see that you are capable.
You can handle failure.
When taking a cold shower, play with hot and cold.
When lifting, play with the feeling of resistance.
Send your ships out into uncharted seas!

\newpage

%bookmark

\begin{center}
\textit{Unlimited possiblities are not suited to man; if they existed, his life would only dissolve in the boundless. - The I Ching or Book of Changes, Hexagram 60.}
\end{center}

\textbf{Obtain an organizing idea}.
An ultimate ideal to play towards.
The more difficult the ideal, the greater one will have to become in order to accomplish it.
It should never be achievable.
It should be an infinite game.
It should not depend on others, only on yourself.
It should be creative.
It should be private.
Art, philosophy.
An organizing idea is determinate optimism.
Everything else may be uncertain.
An organizing idea is logic, everything else is probability.
In order to play, you need to create time and space for play.
Therefore, limits are needed.
The space you carve out for play will be attacked.
Play is life.
Life is play.
To gain money, life must be sacrificed.
To gain power, or any goal, play must be sacrificed.
The goal is always in the future, and life is always here, in this moment.
Resist the siren call of money, popularity, and security.
Your creations don't need the approval of others.
The limits of that space need to be defended.
This choice needs to be made each day anew.
Resist addiction.
Obtain flow.
Speak the truth.
Honor your word.
Not because of duty.
There is no duty.
Never forget that it's a game.
An infinite game, played for fun.

\newpage

\begin{center}
\textit{
Because it's there. - George Mallory
}
\end{center}

\textbf{Magic and dust}.
A flash of lightning in the dark!
You can be grateful you witnessed magic.
Set up the conditions in which magic can be produced.
In isolation and silence, you wait and the muse shall deliver.
Art is the only way to reach truly great states of mind.
Either by appreciating art or by creating it.
The art that we \textit{shall} make will have God as its intended audience, and all other beholders will be merely incidental.
This is how it must be, and how it always has been.
The production of art began with ceremonial objects destined to serve in a cult, and the sheer existence of such objects was always more important than their display.
The elk portrayed by the man of the Stone Age on the walls of his cave was an instrument of magic.
It was meant for the spirits.
Certain sculptures on medieval cathedrals are invisible to the spectator on ground level.
These objects are aimed at Heaven.
Yet at the same time, nothing is special.
The source of magic, and the experience itself, is just dust.
Ashes to ashes.
Even God will look the other way.
This is how it must be, and how it always has been.
There is nothing wrong with it.

\newpage

\begin{center}
\textit{
There's no need to get up to record a thought. If the idea was good, it’ll come back. If it doesn’t come back, it wasn’t that good. - Naval Ravikant
}
\newline
\newline
\textbf{
Completely stop adding things on your todo list.
}
\end{center}

\newpage

\begin{center}
\textit{
Man suffers only because he takes seriously what the gods made for fun. - Alan Watts
}
\end{center}

\textbf{What would you want if you could have anything?}
Suppose that you were able every night to dream any dream you wanted to dream, and that you could, for example, have the power within one night to dream 75 years of time, or any length of time you wanted to have.
And you would, naturally, as you began on this adventure of dreams, you would fulfill all your wishes.
You would have every kind of pleasure you could conceive.
And after several nights of 75 years of total pleasure each you would say “Well that was pretty great. But now let’s have a surprise, let’s have a dream which isn’t under control, where something is gonna happen to me that I don’t know what it's gonna be."
And you would like that and would come out of that and you would say “Wow that was a close shave, wasn’t it?”.
Then you would get more and more adventurous and you would make further and further-out gambles what you would dream.
And finally, you would dream where you are now.
You would dream the dream of living the life that you are actually living today.

\newpage

{\setlength\parindent{0pt}
\textbf{Christopher}: \textit{You ever feel like nothing good was ever gonna happen to you?}
\textbf{Paulie}: \textit{Yeah, and nothing did. So what?}
}
\newline
\newline
Rivers and oceans can be the kings of a hundred valleys.
Because they stay low.
Thus if sages wish to be over people, they must speak humbly to them.
If they wish to be in front of people, they must place themselves behind them.
While alive, the body is soft and pliant.
When dead, it is hard and rigid.
All living things, grass and trees, while alive, are soft and supple; when dead, become dry and brittle.
Thus that which is hard and stiff is the follower of death.
That which is soft and yielding is the follower of life.
Therefore, an inflexible army will not win.
A strong tree will be cut down.
The arrogant, competative, ruthless, judgemental, vengeful, eager to prove themselves...
They occupy a lowly position; while the soft and pliant occupy a higher place.

\newpage

\begin{center}
\textit{
We don’t really know what the good news is and what the bad news is - Kurt Vonnegut
}
\end{center}

\textbf{Maybe}.
Once upon a time there was a Chinese farmer whose horse ran away.
That evening, all of his neighbors came around to commiserate.
They said: \textit{We are so sorry to hear your horse has run away. This is most unfortunate.}
The farmer said: \textit{Maybe.}
The next day the horse came back bringing seven wild horses with it, and in the evening everybody came back and said: \textit{Oh, isn’t that lucky. What a great turn of events. You now have eight horses!}
The farmer again said: \textit{Maybe.}
The following day his son tried to break one of the horses, and while riding it, he was thrown and broke his leg.
The neighbors then said: \textit{Oh dear, that’s too bad} and the farmer responded: \textit{Maybe.}
The next day the conscription officers came around to conscript people into the army, and they rejected his son because he had a broken leg.
Again all the neighbors came around and said: \textit{Isn’t that great!}
Again, he said: \textit{Maybe.}

% removal of options leads to happiness, settling...
% Thank you, that would be enough. - Meister Eckhart

\newpage

\begin{center}
\textit{
I want you to remember that no bastard ever won a war by dying for his country. He won it by making the other poor, dumb bastard die for his country. - George S. Patton, movie Patton
}
\end{center}

\textbf{Keep your identity small}.
We have a capacity to identify strongly with a particular task or mission.
We invest so much of ourselves into an effort that we are unable to separate ourselves from it.
The failure of the effort becomes identical, in our mind, with our own terminal loss of identity.
Cutting your losses becomes cutting your own throat.
We become unable to separate losing from being wrong, being wrong from social death, and social death from actual death.
If you're not person winning this hill, what are you? Well, you are a not dead yet person, which is everything, even if it seems like nothing in the moment.
Notice when you're hitting diminishing returns.
Know your limits and make sure you never cross the thin red line that separates indefinite survivability from an irretrievable spiral of self-destruction.

\newpage

\begin{center}
\textit{
This is water. - David Foster Wallace
}.
\end{center}

\textbf{Everybody worships.} The only choice we get is what to worship.
And the compelling reason for choosing some spiritual thing to worship is that anything else you worship will eat you alive.
If you worship money, then you will never feel you have enough.
Worship your body and beauty and you will always feel ugly.
On one level, we all know this stuff already.
The trick is keeping the truth up front in daily consciousness.
Worship power, you will end up feeling weak and afraid, and you will need ever more power over others to numb you to your own fear.
Worship your intellect, being seen as smart, you will end up feeling stupid, a fraud, always on the verge of being found out.
But the insidious thing about these forms of worship is not that they’re evil or sinful, it’s that they’re unconscious.
They are default settings.
And the so-called real world will not discourage you from operating on your default settings, because the so-called real world of men and money and power hums merrily along in a pool of fear and anger and frustration and craving and worship of self.
The really important kind of freedom involves attention and awareness and discipline, and being able to choose what to think about.
The alternative is unconsciousness, the default setting, the rat race, the constant gnawing sense of having had, and lost, some infinite thing.

\newpage

\begin{center}
\textit{
Life is a series of natural and spontaneous changes. Don't resist them; that only creates sorrow. Let reality be reality. Let things flow naturally forward in whatever way they like. - Lao Tzu
}
\end{center}

\textbf{You are a process of change}.
Awareness of the immediate moment-to-moment passing of the world, the ever-changing existence, the fragility of our own being, and the relative unimportance of the personality we think we are, is a terrifying experience.
It feels like dying.
Instead of avoiding the trauma of the realization, go into it.
Your cells renew themselves periodically, old concepts are replaced by new and your new being reflects life’s experience.
Flow with the changes.
Be that new being.
You don’t have to make the person-you-used-to-be happy.
Adapt.
Be like water.
The only rule to life is that there are no rules.
Invent yourself and reinvent yourself.
Life is a continuing process of change.
Resistance to change and personal growth is pain.
Flow with the changes that are happening to you and enjoy the unfolding of your own life.

%How much of me has already died
%There were whole worlds
%Of colors, sounds and scents
%Unique, elaborate, scenes and emotions
%Some real, some only in my mind’s eye
%Immensity, immensity, immensity...

% even family can abandon you
% only god will always be there for you

% anti-guilt:
% it's fascinanting how, in the moments of catastrophe, there is almost no control over what's happening, there are almost no thoughts, it's like everything is on auto-pilot

% when some emotion is overwhelming, remember that it too will pass...

\end{document}
